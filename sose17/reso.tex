\documentclass[compress,aspectratio=169]{beamer}

%presentation layout

\mode<presentation>
{
  \usetheme{Berlin}
  % \usecolortheme{dove}
  \setbeamercolor{structure}{bg=black,fg=white}
  \setbeamercolor{normal text}{bg=black,fg=white}
  \setbeamercolor{titlepage}{bg=black,fg=white}
  \setbeamercolor{titlelike}{bg=black,fg=white}
  \setbeamercolor{palette primary}{bg=black}
  \setbeamercolor{palette secondary}{bg=black, fg=gray}
  \setbeamercolor{palette tertiary}{bg=black, fg=gray}
  \setbeamercolor{palette quarternary}{bg=black}
  \setbeamercovered{transparent}
  \useinnertheme{rectangles}
  %\usefonttheme{serif}
}

\setbeamertemplate{navigation symbols}{}

%loading packages
\usepackage[ngerman]{babel}
\usepackage[T1]{fontenc}
\usepackage[utf8]{inputenc}
\usepackage{graphicx}
\usepackage{amsmath}
\usepackage{framed}

% vorgeplaenkel
\title[Abschlussplenum SoSe17]{Abschlussplenum der Sommer-ZaPF 2017}

\author{Redeleitung: Björn Guth, Jörg Behrmann}

\institute[Zusammenkunft aller Physikfachschaften]

\date{28. Mai 2017}

\begin{document}

\begin{frame}[plain]{}
  \titlepage
\end{frame}

\section{Formalia}
\begin{frame}{Formalia}
	\begin{enumerate}
		\item Wahl der Redeleitung
		\item Wahl der Protokollanten
		\item Festellung der Beschlussfähigkeit
		\item Wahl der Tagesordnung
		\item Satzungsänderung
		\item GO-Änderung
		\item Anträge
		\item Sonstiges
	\end{enumerate}
\end{frame}

\section{Beschlüsse}
\subsection{Einführung eines Senats}
\begin{frame}{Satzungsänderung Senat}
	\textbf{Füge nach §5 (a) ein:}\\
	Das Plenum ist dem Senat rechenschaftspflichtig.\\
	\textit{(b) Der Senat der ZaPF}\\
	Der Senat ist das ulitmativ höchste Gremium der ZaPF.\\
	Mitglieder werden durch eine dafür auf Beschluss des Senats eingesetzte
	Findungskommission auf Lebenszeit ernannt. Beschlüsse der ZaPF und des
	StAPF müssen vor inkrafttreten durch den Senat bestätigt werden. Der Senat
	ist allen anderen Oragnen der ZaPF gegenüber weisungsbefugt.
	Entscheidungen des Senats müssen durch den StAPF auf Vellum festgehalten
	werden.\\
	Der Sanat tagt mindestens während des ZaPF-Plnums in einem räumlich nahen,
	aber über das Plenum erhöhten Raum. Weiter Sitzungen können nach belieben
	einberufen werden, bedürfen aber keiner öffentlichen Ankündigung.\\
	Während der Sitzung des Sentas herrscht für Senator*innen Togapflicht!
	\hfill Antragsteller: BJörg (RWTFUB)
\end{frame}

\begin{frame}{GO-Änderung Senat}
	\textbf{Füge folgende GO-Anträge in die GO ein:}\\
	\begin{itemize}
		\item zur verschiebung eines Beschlusses in den Senat (auch bekannt
			als "bitte nehmt uns diese schwere Entscheidung ab, geliebte
			Senator*innen!")
		\item zur Überprüfung einer Beschlussvorlage auf Satzungskonformität
			(auch bekannt als "maybe to be continued...? o\_O")
	\end{itemize}
	\hfill Antragsteller: BJörg (RWTFUB)
\end{frame}

\begin{frame}{Einsetzen der ersten Findungskommission für einen Senat}
	\textbf{Die ZaPF möge beschließen:}\\
	Die ZaPF setzt folgende Personen als erste Findungskommission für einen
	Senat ein:
	\begin{itemize}
		\item BJörg (RWTFUB)
		\item Tobi (Düsseldorf)
		\item Marget (FFM)
		\item Mike (die Uni aus der Stadt, die es nicht gibt)
	\end{itemize}
	Die Findungskommissions löst sich mit der einigung auf einen ersten Senat
	automatisch auf.
	\hfill Antragsteller: BJörg (RWTFUB)
\end{frame}

\section{Resolutionen}
\begin{frame}{Resolution für mehr Atombomben}
	\textbf{Die ZaPF möge beschließen:}\\
	Die ZaPF spricht sich für eine bedingungs- und alternativlose nukleare
	Aufrüstung Deutschlands aus. Um gegen studierende Terroristikons effektiv
	vorgehen zu können, müssen diese vor allem in der Nähe von
	Hochschulstandorten stationiert werden. Wir fordern des weiteren einen
	präventiven nuklearen Erstschlag gegen BuFaTas, die dem Allmachtsanspruch
	der ZaPF wiedersprechen.
	\textbf{Begründung:}\\
	Seit Ende des kalten Kriegs ist die Welt in ein Klima des allgegenwärtigen
	Terrorismus abgedriftet. Um dieser globalen Kriese trotzen zu können, muss
	der Staat nicht nur dazu in der Lage sein, jedes Bürgikon
	verdachtsunabhängig als Terroristikon zu identifizieren, sondern auf auch
	gegen solch ungewollte Personen vorzugehen. Nach Auffassung der ZaPF ist
	dies nur durch takische Nuklearwaffen zu gewährleisten.
	\hfill Antragsteller: BJörg (RWTFUB)
\end{frame}

\section{Sonstiges}
\begin{frame}{Sonstiges}
\end{frame}

\end{document}
