\documentclass[compress,aspectratio=169]{beamer}

%presentation layout

\mode<presentation>
{
  \usetheme{Berlin}
  % \usecolortheme{dove}
  \setbeamercolor{structure}{bg=black,fg=white}
  \setbeamercolor{normal text}{bg=black,fg=white}
  \setbeamercolor{titlepage}{bg=black,fg=white}
  \setbeamercolor{titlelike}{bg=black,fg=white}
  \setbeamercolor{palette primary}{bg=black}
  \setbeamercolor{palette secondary}{bg=black, fg=gray}
  \setbeamercolor{palette tertiary}{bg=black, fg=gray}
  \setbeamercolor{palette quarternary}{bg=black}
  \setbeamercovered{transparent}
  \useinnertheme{rectangles}
  %\usefonttheme{serif}
}

\setbeamertemplate{navigation symbols}{}

%loading packages
\usepackage[ngerman]{babel}
\usepackage[T1]{fontenc}
\usepackage[utf8]{inputenc}
\usepackage{graphicx}
\usepackage{amsmath}
\usepackage{framed}

% vorgeplaenkel
\title[Abschlussplenum WiSe17]{Abschlussplenum der Winter-ZaPF 2017}

\author{Redeleitung: Björn Guth, Jörg Behrmann}

\institute[Zusammenkunft aller Physikfachschaften]

\date{31. Oktober 2017}

\begin{document}

\begin{frame}[plain]{}
  \titlepage
\end{frame}

\section{Formalia}
\begin{frame}{Formalia}
	\begin{enumerate}
		\item Wahl der Redeleitung
		\item Wahl der Protokollanten
		\item Festellung der Beschlussfähigkeit
		\item Wahl der Tagesordnung
		\item Anträge
		\item Sonstiges
	\end{enumerate}
\end{frame}

\section{Resolutionen}
\begin{frame}{Resolution für mehr Zwangsexmatrikulationen}
	\textbf{Die ZaPF möge beschließen:}\\
	Deutschland wird zur Zeit von einer noch nie da gewesenen Welle
	ungebildeter Horden überrannt. Um die Qualität des Physikabschlusses
	möglichst hoch zu halten oder nach Möglichkeit sogar noch zu steigern,
	sollen in hierfür angemessenem Maße Studentika zwangsexmatrikliert werden.
	Zur Erfindung der Vorwände lassen wir den Hochschulen dabei freie Hand.

	\textbf{Begründung:}\\
	Überlassen wir den Lesika als Hausarbeit. Bei Nichterfüllen droht
	Zwangsexmatrikulation!

	\hfill Antragsteller: BJörg (RWTFUB)
\end{frame}

\begin{frame}{Resolution für mehr Selbst-/Fremdreflexion}
	\textbf{Die ZaPF möge beschließen:}\\
	Um die Reflexion der ZaPFika zu fördern, fordern wir mehr Reflektoren auf
	Tagungs-T-Shirts.

	\textbf{Begründung:}\\
	Zum Erkenntnisgewinn ist Reflexion nötig und es gibt keine Reflexion ohne
	Reflektoren!

	\hfill Antragsteller: BJörg (RWTFUB)
\end{frame}

\section{Sonstiges}
\begin{frame}{Sonstiges}
\end{frame}

\end{document}
