\documentclass[compress, aspectratio=169]{beamer}

%presentation layout

\mode<presentation>
{
  \usetheme{Berlin}
  % \usecolortheme{dove}
  \setbeamercolor{structure}{bg=black,fg=white}
  \setbeamercolor{normal text}{bg=black,fg=white}
  \setbeamercolor{titlepage}{bg=black,fg=white}
  \setbeamercolor{titlelike}{bg=black,fg=white}
  \setbeamercolor{palette primary}{bg=black}
  \setbeamercolor{palette secondary}{bg=black, fg=gray}
  \setbeamercolor{palette tertiary}{bg=black, fg=gray}
  \setbeamercolor{palette quarternary}{bg=black}
  \setbeamercovered{transparent}
  \useinnertheme{rectangles}
  %\usefonttheme{serif}
}

\setbeamertemplate{navigation symbols}{}

%loading packages
\usepackage[ngerman]{babel}
\usepackage[T1]{fontenc}
\usepackage[utf8]{inputenc}
\usepackage{graphicx}
\usepackage{amsmath}
\usepackage{framed}

% vorgeplaenkel
\title[Abschlussplenum SoSe16]{Abschlussplenum der Sommer-ZaPF 2016}

\author{Björn Guth, Jörg Behrmann}

\institute[Zusammenkunft aller Physikfachschaften]

\date{05. Mai 2016}

\begin{document}

\begin{frame}[plain]{}
  \titlepage
\end{frame}

\section{Formalia}
\begin{frame}{Formalia}
	\begin{itemize}
		\item Wahl der Redeleitung
		\item Wahl der Tagesordnung
	\end{itemize}
\end{frame}

\section{Beschlüsse}
\subsection{Make ZaPF safe again!}
\begin{frame}{Resolution "Make ZaPF safe again!"}
	\textbf{Die ZaPF möge beschließen:}\\
	Die Vertrauenspersonen werden abgeschafft. Im Gegenzug wird der Teilnehmerbeitrag um zehn Euro erhöht. Von diesem Geld sollen alle Frauen mit Pfefferspray und rosa Vergewaltigungsverhinderungstrillerpfeifen ausgestattet werden.\\
	\textbf{Begründung:}\\
	Das Experiment der Vertauenspersonen ist gescheitert!\\
	Es trägt in keinster Weise dazu bei, bestehende Probleme zu lösen. Dies wird vor allem durch den eklatanten Mangel an belastbaren Daten zur Nützlichkeit dieses Bürokratiemonstrums zur nicht zu leugnenden Realität.\\
	\hfill Antragsteller: $\Phi{}$gida
\end{frame}

\subsection{Verpflichtende Frauenquote für alle ZaPF-Gremien}
\begin{frame}{Verpflichtende Frauenquote für alle ZaPF-Gremien}
	\textbf{Die ZaPF möge beschließen:}\\
	In jedem ZaPF-Gremium muss eine Frauenquote von fünfzig von hundert erfüllt sein. Kann diese Quote nicht erreicht werden, dürfen maximal eine gleiche Anzahl Männer wie Frauen kandieren. Weiterhin müssen für jeden Mann auch eine Frau in den studentischen Akkreditierungspool entsandt werden.\\
	\textbf{Begründung:}\\
	Die Unterrepräsentation von Frauen in Positionen der Macht ist das prägende Problem unserer maskulinistischen Gesellschaft. Einzig eine harte Quotierung von Positionen kann dieser himmelschreienden Ungerechtigkeit entgegenwirken.\\
	\hfill Antragsteller: Alice Schwarzer
\end{frame}

\section{Sonstiges}
\begin{frame}{Sonstiges}
\end{frame}

\end{document}
